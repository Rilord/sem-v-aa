\chapter{Технологический раздел}
\label{cha:impl}

В данном разделе приведены требования к программному обеспечению, средства реализации и листинги кода. 

\section{Средства реализации}
Языком реализации данного проекта является C++ стандарта 2017 года. Для графики использовалась библиотека Vulkan API, а библиотека GLM для математических вычислений. 

\section{Требования к программному обеспечению}
Входнымы данными являются:
\begin{itemize}
    \item количество потоков, на которых будет работать программа, количество моделей и сама модель;
\end{itemize}

\section{Реализации алгоритмов}

\begin{lstlisting}[caption=Обычный рендеринг]
	VkCommandBufferBeginInfo cmdBufInfo = commandBufferBeginInfo();

		for (int32_t i = 0; i < drawCmdBuffers.size(); ++i)
		{
			VK_CHECK_RESULT(vkBeginCommandBuffer(drawCmdBuffers[i], &cmdBufInfo));

			VkRenderingAttachmentInfoKHR colorAttachment{};
			colorAttachment.sType = VK_STRUCTURE_TYPE_RENDERING_ATTACHMENT_INFO_KHR;
			colorAttachment.imageView = swapChain.buffers[i].view;
			colorAttachment.imageLayout = VK_IMAGE_LAYOUT_ATTACHMENT_OPTIMAL_KHR;
			colorAttachment.loadOp = VK_ATTACHMENT_LOAD_OP_CLEAR;
			colorAttachment.storeOp = VK_ATTACHMENT_STORE_OP_STORE;
			colorAttachment.clearValue.color = { 0.0f,0.0f,0.0f,0.0f };

			VkRenderingAttachmentInfoKHR depthStencilAttachment{};
			depthStencilAttachment.sType = VK_STRUCTURE_TYPE_RENDERING_ATTACHMENT_INFO_KHR;
			\end{lstlisting}
\begin{lstlisting}[caption=Обычный рендеринг]
			depthStencilAttachment.imageView = depthStencil.view;
			depthStencilAttachment.imageLayout = VK_IMAGE_LAYOUT_DEPTH_ATTACHMENT_OPTIMAL_KHR;
			depthStencilAttachment.loadOp = VK_ATTACHMENT_LOAD_OP_CLEAR;
			depthStencilAttachment.storeOp = VK_ATTACHMENT_STORE_OP_STORE;

			depthStencilAttachment.clearValue.depthStencil = { 1.0f,  0 };

			VkRenderingInfoKHR renderingInfo{};
			renderingInfo.sType = VK_STRUCTURE_TYPE_RENDERING_INFO_KHR;
			renderingInfo.renderArea = { 0, 0, width, height };
			renderingInfo.layerCount = 1;
			renderingInfo.colorAttachmentCount = 1;
			renderingInfo.pColorAttachments = &colorAttachment;
			renderingInfo.pDepthAttachment = &depthStencilAttachment;
			renderingInfo.pStencilAttachment = &depthStencilAttachment;

			vkCmdBeginRenderingKHR(drawCmdBuffers[i], &renderingInfo);

			VkViewport viewport = make_viewport((float)width, (float)height, 0.0f, 1.0f);
			vkCmdSetViewport(drawCmdBuffers[i], 0, 1, &viewport);

			VkRect2D scissor = vks::initializers::rect2D(width, height, 0, 0);
			vkCmdSetScissor(drawCmdBuffers[i], 0, 1, &scissor);

			vkCmdBindDescriptorSets(drawCmdBuffers[i], VK_PIPELINE_BIND_POINT_GRAPHICS, pipelineLayout, 0, 1, &descriptorSet, 0, nullptr);
			vkCmdBindPipeline(drawCmdBuffers[i], VK_PIPELINE_BIND_POINT_GRAPHICS, pipeline);

			model.draw(drawCmdBuffers[i], pipelineLayout);

			vkCmdEndRenderingKHR(drawCmdBuffers[i]);

			VK_CHECK_RESULT(vkEndCommandBuffer(drawCmdBuffers[i]));
		}
	}

	void draw()
	{
	\end{lstlisting}
\begin{lstlisting}[caption=Обычный рендеринг]
		App::prepareFrame();
		submitInfo.commandBufferCount = 1;
		submitInfo.pCommandBuffers = &drawCmdBuffers[currentBuffer];
		VK_CHECK_RESULT(vkQueueSubmit(queue, 1, &submitInfo, VK_NULL_HANDLE));
		VulkanExampleBase::submitFrame();
\end{lstlisting}





\begin{lstlisting}[caption=Многопоточный рендеринг]
void threadRender(uint32_t threadIndex, uint32_t cmdBufferIndex, VkCommandBufferInheritanceInfo inheritanceInfo)
{
	ThreadData *thread = &threadData[threadIndex];
	ObjectData *objectData = &thread->objectData[cmdBufferIndex];


	objectData->visible = frustum.checkSphere(objectData->pos, models.ufo.dimensions.radius * 0.5f);

	if (!objectData->visible)
	{
		return;
	}

	VkCommandBufferBeginInfo cmdBufferBeginInfo = commandBufferBeginInfo();
	commandBufferBeginInfo.flags = VK_COMMAND_BUFFER_USAGE_RENDER_PASS_CONTINUE_BIT;
	cmdBufferBeginInfo.pInheritanceInfo = &inheritanceInfo;

	VkCommandBuffer cmdBuffer = thread->commandBuffer[cmdBufferIndex];

	VK_CHECK_RESULT(vkBeginCommandBuffer(cmdBuffer, &commandBufferBeginInfo));

	VkViewport viewport = make_viewport((float)width, (float)height, 0.0f, 1.0f);
	vkCmdSetViewport(cmdBuffer, 0, 1, &viewport);

	VkRect2D bound = rect2D(width, height, 0, 0);
	vkCmdSetScissor(cmdBuffer, 0, 1, &bound);

	vkCmdBindPipeline(cmdBuffer, VK_PIPELINE_BIND_POINT_GRAPHICS, pipelines.phong);
		\end{lstlisting}
\begin{lstlisting}[caption=Параллельный рендеринг]

	if (!paused) {
		objectData->rotation.y += 2.5f * objectData->rotationSpeed * frameTimer;
		if (objectData->rotation.y > 360.0f) {
			objectData->rotation.y -= 360.0f;
		}
		objectData->deltaT += 0.15f * frameTimer;
		if (objectData->deltaT > 1.0f)
			objectData->deltaT -= 1.0f;
		objectData->pos.y = sin(glm::radians(objectData->deltaT * 360.0f)) * 2.5f;
	}

	objectData->model = glm::translate(glm::mat4(1.0f), objectData->pos);
	objectData->model = glm::rotate(objectData->model, -sinf(glm::radians(objectData->deltaT * 360.0f)) * 0.25f, glm::vec3(objectData->rotationDir, 0.0f, 0.0f));
	objectData->model = glm::rotate(objectData->model, glm::radians(objectData->rotation.y), glm::vec3(0.0f, objectData->rotationDir, 0.0f));
	objectData->model = glm::rotate(objectData->model, glm::radians(objectData->deltaT * 360.0f), glm::vec3(0.0f, objectData->rotationDir, 0.0f));
	objectData->model = glm::scale(objectData->model, glm::vec3(objectData->scale));

	thread->pushConstBlock[cmdBufferIndex].mvp = matrices.projection * matrices.view * objectData->model;

	vkCmdPushConstants(
		cmdBuffer,
		pipelineLayout,
		VK_SHADER_STAGE_VERTEX_BIT,
		0,
		sizeof(ThreadPushConstantBlock),
		&thread->pushConstBlock[cmdBufferIndex]);

	VkDeviceSize offsets[1] = { 0 };
	vkCmdBindVertexBuffers(cmdBuffer, 0, 1, &models.cube.vertices.buffer, offsets);
	vkCmdBindIndexBuffer(cmdBuffer, models.cube.indices.buffer, 0, VK_INDEX_TYPE_UINT32);
\end{lstlisting}
\begin{lstlisting}[caption=Параллельный рендеринг]
	vkCmdDrawIndexed(cmdBuffer, models.cube.indices.count, 1, 0, 0, 0);

	VK_CHECK_RESULT(vkEndCommandBuffer(cmdBuffer));
}
\end{lstlisting}
 
\section{Вывод}

В данном разделе были разработаны исходный код двух алгоритмо

%%% Local Variables:
%%% mode: latex
%%% TeX-master: "rpz"
%%% End:
