\chapter{Аналитический раздел}
\label{cha:analysis}

В данном разделе будут пересказана теоретическая часть, касающаюся алгоритмов рендеринга.

В компьютерной графике рендеринг --- это процесс, в котором абстрактное описание сцены преобразуется в изображение. Пусть сцена является коллекцией геометрически заданных объектов в трехмерном пространестве, где заданы параметры света и точки обзор (виртуальной камеры). Процесс рендеринга высвечивает объекты и отображает их на двумерное пространство изображения, где интенсивность каждого пикселя рассчитаны для полуения финального изображения \cite{crockett1997introduction}.

\section{Рендеринг изображения}
Стандарьный алгоритм интенсивно ипользует понятие потока (pipeline). Стадии потока выполняются параллельно, каждая стадия зависит от предудущей. В нее включены стадии приложения, обработки геометрии, растеризации и обработки пикселя. Стадия приложения практически всегда запущено на ЦПУ общего назначения; обработка геометрии может быть выполена как на стороне ЦПУ, так и на стороне графического ускорителя. Остальные стадии, в большинстве профессиональных решений, происходят на стороне графического ускорителя \cite{akenine2019real}. 

\section{Растеризация полигонов}
Наиболее применимым способом рендеринга является полигональная растеризация. Алгоритм растеризации должен найти все точки (пиксели) внутри полигона, и по заданным параметрам окраски (интенсивность цвета в точках, текстура) закрасить все попвшие внутрь пиксели.

\section{Параллельные рендеринг}
В виду того, что каждая стадия рендеринга зависит от предыдущей, и в первой стадии приложения ЦПУ обрабатывает данные для передачи их видеокарте, то это позволяет паралельно вычислять геометрию для независимых частей сцены.

В Vulkan API \cite{vulkan_2021} реализация многопоточность предоставляется через интерфейс буферов комманд VkCommandPool. Все операции, которые девайс должен выполнить передаются через этот буфер. Преимуществом такого метода является то, что необходимую работу можно выполнить заранее и в несколько потоков \cite{sellers2016vulkan}.


