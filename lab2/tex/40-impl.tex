\chapter{Технологический раздел}
\label{cha:impl}


\section{Средства реализации}
Основным средством реализации в данной лабораторной работе является язык программирования C++[2] ввиду своей высокой скорости работы и большого инструментария библиотек для анализирования программного обеспечения. Программное обеспечение было реализована при поомощи структурного программирования.

\section{Реализации алгоритмов}

\begin{lstlisting}[caption=итеративный Алгоритм Левенштейна]
int edit_distance(char *s, uint32_t ls, char *t, uint32_t lt) {
    int a, b, c;

    if (!ls) return lt;
    if (!lt) return ls;

    if (s[ls - 1] == t[lt - 1])
        return edit_distance(s, ls - 1, t, lt - 1);

    a = edit_distance(s, ls - 1, t, lt - 1);
    b = edit_distance(s, ls, t, lt - 1);
    c = edit_distance(s, ls - 1, t, lt);

    return a + 1;
}

\end{lstlisting}


\begin{lstlisting}[caption=Алгоритм Левенштейна с мемоизацией]
int memoized_edit_distance(char *s, uint32_t ls, char *t, uint32_t lt, int32_t **dp) {

    if (!ls) return lt;
    if (!lt) return ls;

    if (dp[ls - 1][lt - 1] != -1) return dp[ls - 1][lt - 1];

    if (s[ls - 1] == t[lt - 1])
        return dp[ls - 1][lt - 1] = memoized_edit_distance(s, ls - 1, t, lt - 1, dp);
\end{lstlisting}
\begin{lstlisting}[caption=Алгоритм Левенштейна с мемоизацией]

    return dp[ls - 1][lt - 1] = 1 + min3(memoized_edit_distance(s, ls - 1, t, lt - 1, dp),
                                        memoized_edit_distance(s, ls, t, lt - 1, dp),
                                        memoized_edit_distance(s, ls - 1, t, lt - 1, dp));
}
\end{lstlisting}

\begin{lstlisting}[caption=Матрично]
int iterative_levenshtein(char *s, uint32_t ls, char *t, uint32_t lt) {
    uint32_t x, y;
    uint32_t mat[lt + 1][ls + 1];

    mat[0][0] = 0;

    for (x = 1; x <= lt; x++)
        mat[x][0] = mat[x - 1][0] + 1;
    for (y = 1; y <= ls; y++)
        mat[0][y] = mat[0][y - 1] + 1;
    for (x = 1; x <= lt; x++)
        for (y = 1; y <= ls; y++)
            mat[x][y] = min3(
                    mat[x - 1][y] + 1,
                    mat[x][y - 1] + 1,
                    mat[x - 1][y - 1] + (s[y - 1] == t[x - 1] ? 0 : 1)
            );

    return (mat[lt][ls]);
}

}

\end{lstlisting}

\begin{lstlisting}[caption=Функция нахождения расстояния Дамерау--Левенштейна]
int damerau_levenstein(char *s, uint32_t ls, char *t, uint32_t lt) {
    int32_t *dd;
    int32_t i, j, cost, i1, j1, DB;
    int32_t INF = ls + lt;
    int32_t DA[256 * sizeof(int32_t)];
    memset(DA, 0, sizeof(DA));
    if (!(dd = (int32_t *) malloc((ls + 2) * (lt + 2) * sizeof(int32_t)))) {
\end{lstlisting}
\begin{lstlisting}[caption=Функция нахождения расстояния Дамерау--Левенштейна]
        return -1;
    }

    d(0,0) = INF;

    for (i = 0; i < ls + 1; i++) {
        d(i + 1, 1) = i;
        d(i + 1, 0) = INF;
    }

    for (j = 0; j < lt + 1; j++) {
        d(1, j + 1) = j;
        d(0, j + 1) = INF;
    }

    for(i = 1; i < ls + 1; i++) {
        DB = 0;
        for(j = 1; j < lt + 1; j++) {
            i1 = DA[t[j - 1]];
            j1 = DB;
            cost = ((s[i - 1] == t[j - 1])? 0 : 1);
            if(cost == 0) DB = j;
            d(i + 1,j + 1) =
                    min4(d(i,j)+cost,
                         d(i + 1,j) + 1,
                         d(i,j + 1) + 1,
                         d(i1, j1) + (i - i1 - 1) + 1 + (j - j1 - 1));
        }
        DA[s[i-1]] = i;
    }
    cost = d(ls+1,lt+1);
    free(dd);
    return cost;
}
\end{lstlisting}

\section{Тестирование}
В таблице \ref {tab:tabular} приведены тестовые данные

\begin{table}[ht]
  \caption{Пример короткой таблицы с коротким названием}
  \begin{tabular}{|r|c|c|c|l|}
  \hline
  №      & Первое слово & Второе слово  & Ожидаемый результат &  Результат \\
  \hline
  1  &     &     & 0 & 0            \\
    \hline
  2  & abc   & def    & 3 & 3         \\
    \hline
  3  & pool   & loop    & 2 & 2         \\
    \hline
  4  & abbaccb  & ababccb   & 2  & 2         \\
    \hline
  5  &   & aba   & 3  & 3         \\
  \hline
    6  &  gentrification &    & 14  & 14         \\
      \hline
    7  &  joking &  joggling  & 5  & 5         \\
  \hline
  \end{tabular}
  \label{tab:tabular}
\end{table}

\section{Вывод}

В данном разделе были разработаны исходный код всех четырех алгоритмов: вычисления расстояния Левенштейна рекрсивно, с заполнением матрицы, рекурсивно м заполнением матрицы, Дамерау -- Левенштейна с заполнением матрицы. 

%%% Local Variables:
%%% mode: latex
%%% TeX-master: "rpz"
%%% End:
