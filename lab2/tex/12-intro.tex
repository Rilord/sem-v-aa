\tableofcontents{}

\Introduction

Алгоритма Копперсмита---Винограда --- алгоритм умножения матриц, предложенный Доном Копперсмитом и Шмуэлем Виноградом в 1987, который являлся самым асимптотически быстрым с 1990 по 2010 год; его асимптотическая сложность при умнжении две матрицы $n \times n$ в изначальном варианте была равна $O(n^{2.3755})$, где n --- размер стороны матрицы.



Алгоритм на практике используется редко в виду большой константы пропорциональности, для которой действительно эффективно умножение матриц с размерами, несопостваимыми с обьемами современной памяти, и наоборот алгоритм часто применяется как фундамент в других алгоритмах для доказания теоретических временных рамок. 

В реальных условиях крайне часто применяется алгоритм Штрассена ввиду своей простоты и сравнительно невысокой асимптотической сложности $O(n ^{2.807355})$ \cite{stothers2010complexity_abstract}. 

Целью данной лабораторной работы является изучение и реализация следующих алгоритмов: 
\begin{itemize}
    \item стандартный алгоритм умножения матриц;
    \item алгоритм Копперсмита--Винограда;
\end{itemize}



Для достижения поставленной цели необходимо выполнить следующие задачи:
\begin{itemize}
    \item Изучение и реализация трех алгоритмов умножения матриц: обчыный, Копперсмита--Винограда, оптимизированный Копперсмита--Винограда.
    \item Сравнительное описание трудоемкости алгоритмов на основе теоретических расчетов и выбранной модели вычислений.
    \item Сравнительный анализ алгоритмов на основе экспериментальных данных.
\end{itemize}