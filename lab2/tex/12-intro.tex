\tableofcontents{}


\chapter{Введение}
\label{cha:appendix2}

\textbf{Дистанция редактирования Левенштейна} между двумя строками A и B определяется как минимальное количество редактирующих операций, необходимая для превращения A в B или наоборот. 

Великое разнообразие програм имеют надобность в узнавании равенства либо отличия двух строк, и наиболее частые применения сего алгоритма мы можем искать в таких задачах как [1]:
\begin{enumerate}[1)]
    \item коррекция ошибок в слове;
    \item распознование человеческой речи;
    \item сравнение текстовых файлов;
    \item секвенирование ДНК/РНК;
    \item получение и обновление геолокации в программах навигации.
\end{enumerate}

Цель:
изучение метода динамического программирования с использованием материалов о нахождении расстояния Левенштейна и Дамерау-Левенштейна
Задачи лабораторной работы:
\begin{itemize}
    \item Изучение алгоритмов Левенштейна и Дамерау-Левенштейна.
    \item Применение методов динамического программирования для реализации алгоритмов.
    \item Сравнительный анализ алгоритмов на основе экспериментальных данных.
\end{itemize}