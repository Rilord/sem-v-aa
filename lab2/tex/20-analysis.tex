\chapter{Аналитический раздел}
\label{cha:analysis}

В данном разделе будут пересказана теоретическая часть, касающаюся алгоритмов умножения матриц.

\section{Стандартный алгоритм}

Пусть даны две матрицы
\begin{equation} \label{eu_eqn}
A = \begin{bmatrix} 
    a_{11} & a_{12} & \dots \\
    \vdots & \ddots & \\
    a_{l1} &        & a_{lm} 
    \end{bmatrix},
B = \begin{bmatrix} 
a_{11} & a_{12} & \dots \\
\vdots & \ddots & \\
a_{m1} &        & a_{mn} 
\end{bmatrix},
\end{equation}
тогда матрица $C$ будет иметь вид
\begin{equation}
   A = \begin{bmatrix} 
    a_{11} & a_{12} & \dots \\
    \vdots & \ddots & \\
    a_{l1} &        & a_{ln} 
    \end{bmatrix}, 
\end{equation}
где
\begin{equation}
c_{ij} = \sum_{r=1}^{m}a_{ir}b_{rj} (i)
\end{equation}
\cite{stothers2010complexity_simple}

\section{Алгоритм Копперсмита--Винограда}
Каждый эелемнт в результирующей матрице $C$ представляет скалярное произведение соответсвующей строки и столбца исходных матриц.

Рассмотрим два вектора V и W; их скалярное произведение равно. Это уравнение можно выразить как: 
\begin{equation}
\label{eq4}
V \cdot W =(v_1 +w_2)(v_2 +w_1)+(v_3 +w_4)(v_4 +w_3)−v_1v_2 −v_3v_4 −w_1w_2 −w_3w_4
\end{equation}

В \ref{eq4} можно заранее считать части уравнений, что пиведет нас от шести умножений и десяти сложений к двум умножений и пятью сложениями. На ЭВМ алгоритм быстрее стандартного из-зи меньшего числа умножений \cite{williams2012multiplying}.

\section{Вывод}

В данном разделе были рассмотрены два алгоритма умножения матриц: обычная и Копперсмита---Винограда. В дальнейшем изученные материал будет применен в реализации данных алгоритмов. 

Входными данными будут являться две матриц и их размеры, а выходными данными --- их произведение. 

Ограничением на входные данные является:
\begin{itemize}
    \item Количество строк первой матрицы должно быть равно количеству столбцов второй;
    \item Размерности матриц должны быть неотрицательными и целыми числами.
\end{itemize}