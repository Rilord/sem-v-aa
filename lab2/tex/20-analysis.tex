\chapter{Аналитический раздел}
\label{cha:analysis}
%
% % В начале раздела  можно напомнить его цель
%

\textbf{Дистанция редактирования Левенштейна} между двумя строками A и B определяется как минимальное количество редактирующих операций, необходимая для превращения A в B или наоборот. 

Цены операций зависят от вида операции(вставка, удаление, замена) и/или от участвующих в ней символов, отражая разнуя вероятность разных ошибок при вводе текста. Общий случай выглядит так: 
\begin{itemize}
\item $w(a,b)$ -- цена замены символа $a$ на символ $b$.
\item $w(\lambda,b)$ -- цена вставки символа $b$.
\item $w(a, \lambda)$ -- цена удаления символа $a$.
\end{itemize}

\section{Рекурсивный алгоритм нахождения расстояния Левенштейна}

Ресстояние Левенштейна между двумя строками a и b может быть вычислено по формуле \ref{eq:D}, где $|a|$

\begin{equation}
\label{eq:D}
	D(i,j) = \left\{ \begin{array}{ll}
	0, & \textrm{$i = 0, j = 0$}\\
	i, & \textrm{$j = 0, i > 0$}\\
	j, & \textrm{$i = 0, j > 0$}\\
	min(\\
	D(i,j-1)+1,\\
	D(i-1, j) +1, &\textrm{$j>0, i>0$}\\
	D(i-1, j-1) + m(S_{1}[i], S_{2}[j])\\
	),
	\end{array} \right.
\end{equation}

\noindent
где $m(a,b)$ равна нулю, если $a=b$ и единице при ином раскладе; $min\{\,a,b,c\}$ возвращает наименьший из аргументов.


\section{Матричный алгоритм нахождения расстояния Левенштейна}
В целях оптимицзации нахождения расстояния Левенштейна возможно хранение в матрице соответствующих промежуточных значений.

\section{Рекурсивный алгоритм нахождения расстояния Левенштейна с заполнением матрицы}
Рекурсивный метод предпологает паралельное заполнение матрицы с использованием рекурсии. Для необработанных данных -- результат нахождения расстояния заносится в результирующую матрицу; уже обработанные данные игнорируются.

\section{Расстояния Дамерау --- Левенштейна}

Расстояние Дамерау -- Левенштейна следует искать по формуле \ref{eq:DL}, что имеет следующий вид:

\begin{equation}
\label{eq:DL}
\[ D(i, j) =  \left\{
	\begin{aligned}
		  & 0, &   & i = 0, j = 0 \\
		  & i, &   & i > 0, j = 0 \\
		  & j, &   & i = 0, j > 0 \\		    	
		&min \left\{
		\begin{aligned}
		&D(i, j - 1) + 1,\\
		&D(i - 1, j) + 1,\\
		&D(i - 1, j - 1) + m(S_{1}[i], S_{2}[i]), \\
		&D(i - 2, j - 2) + m(S_{1}[i], S_{2}[i]),\\
	\end{aligned} \right.
	&& 
	\begin{aligned}
		  & , \text{ если } i, j > 0         \\
		  & \text{ и } S_{1}[i] = S_{2}[j - 1]  \\
		  & \text{ и } S_{1}[i - 1] =  S_{2}[j] \\
	\end{aligned} \\ 
	&min \left\{
	\begin{aligned}
		  & D(i, j - 1) + 1,                         \\
		  & D(i - 1, j) + 1,                         \\
		  & D(i - 1, j - 1) + m(S_{1}[i], S_{2}[i]), \\
	\end{aligned} \right.  &&, \text{иначе}
	\end{aligned} \right.
\]	
\end{equation}

Увы, подобно разобранному нами рекурсивному методу, ортодоксальное использование вышеуказанной формулы чрезвычайно неплодотворно, чего мы не могли бы утверждать про матричный способ хранения промежуточных значений, ранее описанного нами в 1.3. 

\section{Вывод}
В данном разделе были рассмотрены алгоритмы нахождения расстояния Левенштейна и Дамерау-Левенштейна, который является модификаций первого, учитывающего возмож- ность перестановки соседних символов. Формулы Левенштейна и Дамерау – Левенштейна для рассчета расстояния между строками задаются рекурсивно, а следовательно, алгоритмы могут быть реализованы рекурсивно или итерационно.