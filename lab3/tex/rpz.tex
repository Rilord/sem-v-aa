%% Преамбула TeX-файла

% 1. Стиль и язык
\documentclass[utf8x, 14pt]{G7-32} % Стиль (по умолчанию будет 14pt)
\include{preamble.inc}

% Настройки листингов.
\include{listings.inc}

\usepackage{graphicx}
\graphicspath{{inc/}}
\usepackage{tabularx} % in the preamble
\usepackage[normalem]{ulem}

% Полезные макросы листингов.
\include{macros.inc}


% Стиль титульного листа и заголовки


\begin{document}

\include{00-title}

\frontmatter % выключает нумерацию ВСЕГО; здесь начинаются ненумерованные главы: реферат, введение, глоссарий, сокращения и прочее.
\tableofcontents{}


\chapter{Введение}
\label{cha:appendix2}

В данной лабораторной работе будeт рассмотрены различные методы сортировок. Алгоритмы сортировки имеют широкое практическое применение. Сортировки используются в большом спектре задач, включая обработку коммерческих, сейсмических, космических данных. Часто сортировка является просто вспомогательной операцией для упорядочивания данных, упрощения последующих алгебраических действий над данными. 

Сортировка применяется во многих областях программирования, например, базы данных или математические программы. Упорядоченные объекты содержатся в телефонных книгах, ведомостях налогов, в библиотеках, в оглавлениях, в словарях. 

В некоторых вычислительных системах на сортировку уходит больше половины машин- ного времени. Исходя из этого, можно заключить, что либо сортировка имеет много важных применений, либо ею часто пользуются без нужды, либо применяются в основном неэффективные алгоритмы сортировки. В настоящее время, в связи с экспоненциально возросшими объемами данных, вопрос эффективной сортировки данных снова стал актуальным. 
В настоящее время в сети Интернет можно найти результаты производительности алгоритмов сортировки для ряда ведущих центров данных. При этом используются различные критерии оценки эффективности.

Целью данной лабораторной работы является изучение и реализация следующих алгоритмов: 
\begin{itemize}
    \item сортировки пузырьком;
    \item пирамидальной сортировки
    \item поразрядной сортировки.
\end{itemize}

Для достижения поставленной цели необходимо выполнить следующие задачи: 
\begin{itemize}
    \item рассмотреть и изучить поразряжной сортировки, пирамидальной сортировки и сортировки пузырьком.
    \item Разработать реализацию каждой из этих соритровок.
    \item Теоретически оценить их трудоемоксть.
    \item Сравнить временные характеристики экспериментально.
\end{itemize}



%\listoffigures                         % Список рисунков

%\listoftables                          % Список таблиц

%\NormRefs % Нормативные ссылки 
% Команды \breakingbeforechapters и \nonbreakingbeforechapters
% управляют разрывом страницы перед главами.
% По-умолчанию страница разрывается.

% \nobreakingbeforechapters
% \breakingbeforechapters


\mainmatter % это включает нумерацию глав и секций в документе ниже

\chapter{Аналитический раздел}
\label{cha:analysis}

В данном разделе будут пересказана теоретическая часть, касающаюся алгоритмов умножения матриц.

\section{Стандартный алгоритм}

Пусть даны две матрицы
\begin{equation} \label{eu_eqn}
A = \begin{bmatrix} 
    a_{11} & a_{12} & \dots \\
    \vdots & \ddots & \\
    a_{l1} &        & a_{lm} 
    \end{bmatrix},
B = \begin{bmatrix} 
a_{11} & a_{12} & \dots \\
\vdots & \ddots & \\
a_{m1} &        & a_{mn} 
\end{bmatrix},
\end{equation}
тогда матрица $C$ будет иметь вид
\begin{equation}
   A = \begin{bmatrix} 
    a_{11} & a_{12} & \dots \\
    \vdots & \ddots & \\
    a_{l1} &        & a_{ln} 
    \end{bmatrix}, 
\end{equation}
где
\begin{equation}
c_{ij} = \sum_{r=1}^{m}a_{ir}b_{rj} (i)
\end{equation}
\cite{stothers2010complexity_simple}

\section{Алгоритм Копперсмита--Винограда}
Каждый эелемнт в результирующей матрице $C$ представляет скалярное произведение соответсвующей строки и столбца исходных матриц.

Рассмотрим два вектора V и W; их скалярное произведение равно. Это уравнение можно выразить как: 
\begin{equation}
\label{eq4}
V \cdot W =(v_1 +w_2)(v_2 +w_1)+(v_3 +w_4)(v_4 +w_3)−v_1v_2 −v_3v_4 −w_1w_2 −w_3w_4
\end{equation}

В \ref{eq4} можно заранее считать части уравнений, что пиведет нас от шести умножений и десяти сложений к двум умножений и пятью сложениями. На ЭВМ алгоритм быстрее стандартного из-зи меньшего числа умножений \cite{williams2012multiplying}.

\section{Вывод}

В данном разделе были рассмотрены два алгоритма умножения матриц: обычная и Копперсмита---Винограда. В дальнейшем изученные материал будет применен в реализации данных алгоритмов. 

Входными данными будут являться две матриц и их размеры, а выходными данными --- их произведение. 

Ограничением на входные данные является:
\begin{itemize}
    \item Количество строк первой матрицы должно быть равно количеству столбцов второй;
    \item Размерности матриц должны быть неотрицательными и целыми числами.
\end{itemize}
\chapter{Конструкторский раздел}
\label{cha:design}

В данном разделе будут рассмотрены схемы алгоритмов и структура реализации.

\section{Схемы алгоритмов сортировки}
На рисунке ~\ref{fig:radix_sort} приведена схема алгоритма поразрядной сортировки.
На рисунке ~\ref{fig:heap_sort} приведена схема пирамидальной сортировки.
На рисунке ~\ref{fig:bubble_sort} приведена схема пузырькового алгоритма сортировки.

\begin{figure}
    \centering
    \includegraphics[height=0.75\textheight]{sem-v-aa-master/lab1/tex/inc/schemes/radix_sort.png}
    \caption{Схема алгоритма поразрядной сортировки}
    \label{fig:radix_sort}
\end{figure}

\begin{figure}
    \centering
    \includegraphics[height=0.75\textheight]{sem-v-aa-master/lab1/tex/inc/schemes/bubble_sort.png}
    \caption{Схема алгоритма пузырьковой сортировки}
    \label{fig:bubble_sort}
\end{figure}

\begin{figure}
    \centering
    \includegraphics[height=0.75\textheight]{sem-v-aa-master/lab1/tex/inc/schemes/heap_sort.png}
    \caption{Схема алгоритма пирмаидальной сортировки}
    \label{fig:heap_sort}
\end{figure}

\section{Модель вычислений} 
Для последующего вычисления трудоемкости введем модель вычислений:
\begin{enumerate}[1.]
    \item Операции из списка \ref{eq:operations} имеют трудоемкость $1$. 
    \begin{equation}
        +,-,/,\%,=,\ne,<,>,\leq,\geq,[\;],++,--
        \label{operations}
    \end{equation}
    \item Трудоемкость условного оператора расчитвыается как \ref{eq:conditional}
    \begin{equation}
        f_{if} = f_{условия} + 
        \begin{cases}
            f_{A},& \text{если условие выполняется},\\
            f_{B},& \text{иначе}.
        \end{cases}
        \label{eq:conditional}
    \end{equation}
    \item Трудоемкость цикла рассчитывается как \ref{eq:loop}
    \begin{equation}
        f_{\text{цикл}} = f_{\text{сравнения}} + N \cdot (f_{\text{тела}} + f_{\text{инкремента}} + f_{\text{сравнения}})
        \label{eq:loop}
    \end{equation}
    \item трудоемкость вызова функции равно $0$.
\end{enumerate}

\section{Трудоемкость алгоритмов}
Пусть размер массив будет обозначаться как $N$.
\subsection{Алгоритм сортировки пузырьком}

Трудоемкость алгоритма сортировки пузырьком состоит из:
\begin{itemize}
    \item трудоемкость сравнения и инкремента внешнего цикла $i \in [1\dots N)$ \ref{eq:outer_loop} \begin{equation}
        f_i = 2 + 2(N - 1)
        \label{eq:outer_loop}
    \end{equation}
    суммарная трудоескость внутренних циклов, количество итераций которых меняется в промежутке $[2\dots N - 1]$ \ref{eq:inner_loop} \begin{equation}
        f_j = 3(N - 1) + \dfrac{N^2 - N}{2} \cdot(3 + f_{ij})
        \label{eq:inner_loop}
    \end{equation}
    \item трудоемкость условия во внутреннем цикле \ref{eq:inner_condition} \begin{equation}
        f_{ij} = 4 +         
        \begin{cases}
            0, & \text{в лучшем случае},\\
            9,& \text{в худшем случае}.
        \end{cases}
        \label{eq:inner_condition}
    \end{equation}
\end{itemize}

Трудоемкость в лучшем случае \ref{eq:bubble_best}
\begin{equation}
    f_{\text{best}} = \frac{7}{2}N^2 + \frac{3}{2}N - 3 \approx \frac{7}{2}N^2 = O(N^2)
    \label{eq:bubble_best}
\end{equation}
Трудоемкость в худшем случае \ref{eq:bubble_worst}
\begin{equation}
    f_{\text{worst}} = 8N^2 + 8N - 3 \approx 8N^2 = O(N^2)
    \label{eq:bubble_worst}
\end{equation}

\subsection{Алгоритм пирамидальной сортировки}

Алгоритм пирамидальной сортировки начинается с построения кучи Build-Max-Heap:

Трудоемкость сравнения и инкремента внешнего цикла $i \in [1..N / 2)$
\begin{equation}
    f_i = 2 + 2(N - 1)
    \label{eq:heap_outer}
\end{equation}

Трудоемкость внутренного цикла, вызванного рекурсивно \ref{eq:heap_inner}, учитывая, что в среднем, длина поддерева будет равна $2/3n$:
    
\begin{equation}
    T(n) \leq T(2n/3) + O(1)
    \label{eq:heap_inner}
\end{equation}

Это можно свести к $O(\log n)$, пользуясь теоремой о рекурентных соотношений \cite{intro_to_algo_heap}.

\subsection{Алгоритм поразрядной сортировки}
Пусть даны N чисел, имеющих разрядность $b$ и любое положительное целое число $r < b$.

Для люого $r \leq b$ мы рассматриваем ключ, имеющим $d = [ b / r ]$ цифр по $r$ бит каждая. Каждая цифра лежит в диапазоне от $0$ до $2^{r} - 1$. В таком случае мы можем использовать сортировку подсчетом с значением $k = 2^r - 1$. Каждый проход сортировки подсчетом займет $O(n + k) = O(n + 2^r)$, а проходов всего --- $d$. Получаем время сортировки $O(d(n + 2^r)) = O((b/r)(n + 2^r))$ \cite{intro_to_algo_radix}.

\subsection{Структуры данных}

В реализации алгоритма пирамидальной сортировки было использовано бинаное дерево. В реализации поразрядной сортировки использовалась карманная сортировка, где использовались несколько массивов (коллекций), в которых содержались значения разрядов ключей массива.

\section{Тестирование}
Тестирование будет проведено на трех классах эквивалентности: заранее отсортированном массиве (в порядке возратания), заранее отсортированном массиве (в порядке убывания) и случайно сформированном массиве.


\section{Вывод}

На основе теоретического материала из аналитического раздела были построены схемы реализаций исследуемых алгоритмов.

%%% Local Variables:
%%% mode: latex
%%% TeX-master: "rpz"
%%% End:

\chapter{Технологический раздел}
\label{cha:impl}


\section{Средства реализации}
Основным средством реализации в данной лабораторной работе является язык программирования C++[2] ввиду своей высокой скорости работы и большого инструментария библиотек для анализирования программного обеспечения. Программное обеспечение было реализована при поомощи структурного программирования.

\section{Реализации алгоритмов}

\begin{lstlisting}[caption=итеративный Алгоритм Левенштейна]
int edit_distance(char *s, uint32_t ls, char *t, uint32_t lt) {
    int a, b, c;

    if (!ls) return lt;
    if (!lt) return ls;

    if (s[ls - 1] == t[lt - 1])
        return edit_distance(s, ls - 1, t, lt - 1);

    a = edit_distance(s, ls - 1, t, lt - 1);
    b = edit_distance(s, ls, t, lt - 1);
    c = edit_distance(s, ls - 1, t, lt);

    return a + 1;
}

\end{lstlisting}


\begin{lstlisting}[caption=Алгоритм Левенштейна с мемоизацией]
int memoized_edit_distance(char *s, uint32_t ls, char *t, uint32_t lt, int32_t **dp) {

    if (!ls) return lt;
    if (!lt) return ls;

    if (dp[ls - 1][lt - 1] != -1) return dp[ls - 1][lt - 1];

    if (s[ls - 1] == t[lt - 1])
        return dp[ls - 1][lt - 1] = memoized_edit_distance(s, ls - 1, t, lt - 1, dp);
\end{lstlisting}
\begin{lstlisting}[caption=Алгоритм Левенштейна с мемоизацией]

    return dp[ls - 1][lt - 1] = 1 + min3(memoized_edit_distance(s, ls - 1, t, lt - 1, dp),
                                        memoized_edit_distance(s, ls, t, lt - 1, dp),
                                        memoized_edit_distance(s, ls - 1, t, lt - 1, dp));
}
\end{lstlisting}

\begin{lstlisting}[caption=Матрично]
int iterative_levenshtein(char *s, uint32_t ls, char *t, uint32_t lt) {
    uint32_t x, y;
    uint32_t mat[lt + 1][ls + 1];

    mat[0][0] = 0;

    for (x = 1; x <= lt; x++)
        mat[x][0] = mat[x - 1][0] + 1;
    for (y = 1; y <= ls; y++)
        mat[0][y] = mat[0][y - 1] + 1;
    for (x = 1; x <= lt; x++)
        for (y = 1; y <= ls; y++)
            mat[x][y] = min3(
                    mat[x - 1][y] + 1,
                    mat[x][y - 1] + 1,
                    mat[x - 1][y - 1] + (s[y - 1] == t[x - 1] ? 0 : 1)
            );

    return (mat[lt][ls]);
}

}

\end{lstlisting}

\begin{lstlisting}[caption=Функция нахождения расстояния Дамерау--Левенштейна]
int damerau_levenstein(char *s, uint32_t ls, char *t, uint32_t lt) {
    int32_t *dd;
    int32_t i, j, cost, i1, j1, DB;
    int32_t INF = ls + lt;
    int32_t DA[256 * sizeof(int32_t)];
    memset(DA, 0, sizeof(DA));
    if (!(dd = (int32_t *) malloc((ls + 2) * (lt + 2) * sizeof(int32_t)))) {
\end{lstlisting}
\begin{lstlisting}[caption=Функция нахождения расстояния Дамерау--Левенштейна]
        return -1;
    }

    d(0,0) = INF;

    for (i = 0; i < ls + 1; i++) {
        d(i + 1, 1) = i;
        d(i + 1, 0) = INF;
    }

    for (j = 0; j < lt + 1; j++) {
        d(1, j + 1) = j;
        d(0, j + 1) = INF;
    }

    for(i = 1; i < ls + 1; i++) {
        DB = 0;
        for(j = 1; j < lt + 1; j++) {
            i1 = DA[t[j - 1]];
            j1 = DB;
            cost = ((s[i - 1] == t[j - 1])? 0 : 1);
            if(cost == 0) DB = j;
            d(i + 1,j + 1) =
                    min4(d(i,j)+cost,
                         d(i + 1,j) + 1,
                         d(i,j + 1) + 1,
                         d(i1, j1) + (i - i1 - 1) + 1 + (j - j1 - 1));
        }
        DA[s[i-1]] = i;
    }
    cost = d(ls+1,lt+1);
    free(dd);
    return cost;
}
\end{lstlisting}

\section{Тестирование}
В таблице \ref {tab:tabular} приведены тестовые данные

\begin{table}[ht]
  \caption{Пример короткой таблицы с коротким названием}
  \begin{tabular}{|r|c|c|c|l|}
  \hline
  №      & Первое слово & Второе слово  & Ожидаемый результат &  Результат \\
  \hline
  1  &     &     & 0 & 0            \\
    \hline
  2  & abc   & def    & 3 & 3         \\
    \hline
  3  & pool   & loop    & 2 & 2         \\
    \hline
  4  & abbaccb  & ababccb   & 2  & 2         \\
    \hline
  5  &   & aba   & 3  & 3         \\
  \hline
    6  &  gentrification &    & 14  & 14         \\
      \hline
    7  &  joking &  joggling  & 5  & 5         \\
  \hline
  \end{tabular}
  \label{tab:tabular}
\end{table}

\section{Вывод}

В данном разделе были разработаны исходный код всех четырех алгоритмов: вычисления расстояния Левенштейна рекрсивно, с заполнением матрицы, рекурсивно м заполнением матрицы, Дамерау -- Левенштейна с заполнением матрицы. 

%%% Local Variables:
%%% mode: latex
%%% TeX-master: "rpz"
%%% End:

\chapter{Исследовательский раздел}
\label{cha:research}

В данном разделе будет проведено функциональное тестирование разработанного программного обеспечения. Также будет проведено измерение каждого из реализованных алгоритмов.

\section{Пример работы}

\section{Технические характеристики}
\begin{itemize}
    \item Процессор: AMD Ryzen 2700 4.00GHz \cite{ryzen}.
    \item Видеокарта: NVIDIA GTX 1080 \cite{gtx1080}.
    \item Оперативная память: 16GiB.
    \item Операционная система: Linux Kernel 5.14.8 \cite{kernel}.
\end{itemize}



\section{Временные характеристики}

Для сравнения была взята трехмерная модель размером 550Кб. В каждом замере выбранная модель клонировалась $N$ раз, и затем в каждом кадре над $N$ моделями проводились одинаковые геометрические преобразования.

Время выполнения алгоритмов было найдено при помощи  стандартной библиотеки языка C++ \cite{iso_2017}. 
Результаты измерения времени (в милисекундах) были зафиксированы в Таблице \ref{tab:benchmark}.

\begin{table}[ht]
  \caption{Замер времени для количества разного объектов (от 2 до 1000) в милисекундах на один кадр. }
  \begin{tabular}{|c|c|c|c|c|c|}
  \hline
  Кол-во объектов & 1 & 2  & 4 & 8 & 16\\
  \hline
  32  &  $0.76$   &  $0.54$   & $0.41$ & 0.36 & 0.40          \\
    \hline
  128  & $1.41$   & $1.08$    & 0.88  &  0.67 & 0.63    \\
    \hline
  512  & $4.02$   & $2.79$    & 2.34 & 1.69 & 1.57        \\
    \hline
  2048  & $13.33$  & $11.90$   & 8.47  & 6.33 & 5.78\\
    \hline
  4096 &  $33.33$ & $20.41$ & 16.95 & 11.90 & 11.24\\
    \hline
  8192  &  $52.63$ & $35.71$ & 32.26 & 21.74 & 20.41\\
    \hline
  \end{tabular}
  \label{tab:benchmark}
\end{table}

Из рисунка \ref{fig:timestamps} видно, что увеличение количества потоков уменьшает количество времени, затраченного на отрисовку одного кадра. Наиболее эффекутивно использование 8 потоков, то есть равному количеству логических ядер ЭВМ. 

Рост производительности при использовании более одного потока минимален при небольшой нагрузке сцены (128 объектов), и существенен при обработке 8192 модели: при 1 потоке --- 52.643 мс/кадр (или 19.2 кадров в секунду) и 20.41 мс/кадр (или 49 кадров в секунду).

При малом количестве объектов (менее 100) использование параллельной реализации рендеринга проигрывает обычной ввиду необходимости выделения дополнительных ресурсов под поточную реализацию. 

\begin{figure}
    \centering
\includegraphics[width=\textwidth]{sem-v-aa-master/lab1/tex/inc/plots/lab4-threads.png}

    \caption{Временные характеристики для различного количества работающих потоков и объектов сцены}
    \label{fig:timestamps}
\end{figure}

\section{Вывод}
В данном разделе было произведено сравнение вышеизложенных алгоритмов.

Наиболее эффективной оказалась параллельная реализация рендеринга, давая прирост производительности более в 2 раза в крайне нагруженной сцене (8192 объекта и их преобразований). Такой прирост был достигнут при использовании 16 потоков.


%%% Local Variables:
%%% mode: latex
%%% TeX-master: "rpz"
%%% End:


\backmatter
            
\Conclusion % заключение к отчёту

В результате выполнения данной лабораторной работы удалось выяснить, что наиболее эффективным из рассмотренных нами алгоритмов является алгоритм Копперсмита--Винограда. Однако увеличение эффективности повлекло за собой увеличение расхода в памяти. 

Как было сказано в аналитическом разделе, эффективность в сранении с простым алгоритмом винограда тем выше, чем больше размерность умножаемых матриц.

Из этого можно сделать вывод, что в универсальной реализации программы перемножения матрицы необходимо использовать простой алгоритм умножения матриц для размерности меньше 200 и алгоритм Копперсмита--Винограда для больших размерностей.

В рамках выполнения данной лабораторной работы были выполнены следующие задачи:
\begin{enumerate}
    \item изучен и реализован простой алгоритм умножения матриц;
    \item изучен и реализован алгоритм Копперсмита---Винограда умножения матриц; 
    \item оптимизирован алгоритм Винограда умножения матриц; 
    \item произведена оцена трудноемкости реализаций алгоритмов умножения матриц; 
    \item произведены сравнения временных характеристик вышеизложенных алгоритмов.
\end{enumerate}

%%% Local Variables: 
%%% mode: latex
%%% TeX-master: "rpz"
%%% End: 
%% заключение


% % Список литературы при помощи BibTeX
% Юзать так:
%
% pdflatex rpz
% bibtex rpz
% pdflatex rpz

\begin{thebibliography}{5}
\bibitem{levenshtein}
Levenshtein Distance Technique in Dictionary Lookup Methods: An Improved Approach, Rishin Haldar and Debajyoti Mukhopadhyay, 2011, https://arxiv.org/abs/1101.1232, С. 4 -- 5.
\bibitem{cppreference}
C++ reference [Электронный ресурс]. Режим доступа: https://en.cppreference.com/w/ (дата обращения 20.10.2021).
\bibitem{linux}
The Linux Kernel Organization[Электронный ресурс] Режим доступа: https://www.kernel.org (дата обращения: 7.10.2021).
\bibitem{google_test}
GoogleTest [Электронный ресурс]. Режим доступа: https://google.github.io/googletest/ (дата обращения 7.10.2021).
\bibitem{Ryzen}
AMD Ryzen™ 7 2700 Processor [Электронный ресурс]. Режим доступа: https://www.amd.com/en/products/cpu/amd-ryzen-7-2700 (дата обращения 7.10.2021).
\end{thebibliography}
%%% Local Variables: 
%%% mode: latex
%%% TeX-master: "rpz"
%%% End: 



\end{document}

%%% Local Variables:
%%% mode: latex
%%% TeX-master: t
%%% End:
