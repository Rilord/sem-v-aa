\chapter{Аналитический раздел}
\label{cha:analysis}
%
% % В начале раздела  можно напомнить его цель
%

\section{Поразрядная сортировка}
Подразрядные алгоритмы сортировки (Radix sort) принимают за числа ключи, в которых каждая цифра --- целочисленное в пределах $\langle 0\dots(m - 1) \rangle$
, где $m$ --- разряд. Разряд часто выбран в пользу уменьшения времени пробега и крайне варируется на количестве ключей и самой имплементации.

Подразрядная сортировка работает, разбивая ключи на числа; далее соритруя одно число за раз, начиная с наименьшего числа \cite{zagha1991radix}.
\section{Пирамидальная сортировка}
Пирамидальная сортировка, --- в англоязычных источниках Heapsort, --- сортирует массив $\text{A} [1\dots N]$ из $\text{N}$ ключей, сначала выбирая наибольший ключ, и меняя его с последним. Второй наибольший ключ затем выбирается из оставшихся, и аналогично меняется местами уже с предпоследним ключом. Действуя ортодоксально, в конечном счете, станет очевидно, что результирующий массив будет содержать исходные ключи в порядке возрастания \cite{schaffer1993analysis}.
\section{Сортировка пузырьком}
Метод сортировки пузырьком, также известный как смежное сравнение, является простым методом сортировки внутреннего обмена. Это процедура, в которой соритровка элементов будет происходить посредством обмена смежных элементов \cite{min2010analysis}. 

\section{Вывод}
В данном разделе были рассмотрены три вида сторировки: поразрядная, пирамидальная и пузырьковая. В дальнейшем изученные материал будет применен в реализации данных алгоритмов. 

Входными данными будут являться массив и его длина, а выходными данными --- отсоритрованный по возрастанию массив. 

Ограничением на входные данные является:
\begin{itemize}
    \item Массив должен представлять последовательность из N однотипных целочисленных ключей;
    \item Размер массива должен быть неотрицательным целым числом.
\end{itemize}

