\chapter{Технологический раздел}
\label{cha:impl}

В данном разделе приведены требования к программному обеспечению, средства реализации и листинги кода. 

\section{Средства реализации}
Основным средством реализации в данной лабораторной работе является язык программирования C++17 \cite{iso_2017} ввиду своей высокой скорости работы и большого инструментария библиотек для анализирования программного обеспечения. Программное обеспечение было реализована при помощи структурного программирования.

\section{Требования к программному обеспечению}
Входнымы данными являются:
\begin{itemize}
    \item размерность массива $\text{size}$;
    \item $\text{size}$ однотипных элементов массива $\text{arr}$.
\end{itemize}
Выходными данными получается отсортированный массив $\text{arr}$.



\section{Реализации алгоритмов}

\begin{lstlisting}[caption=Алгоритм поразрядной сортировки]
void radix_ui(register unsigned int vector[], register const unsigned int size) {

    const int MAX_UINT__ = ((((1 << ((sizeof(unsigned int) << 3) - 2)) - 1) << 1) + 1);
    const int LAST_EXP__ = (sizeof(unsigned int) - 1) << 3;
    
#define PRELIMINARY__ 100
#define MISSING_BITS__ exp < (sizeof(unsigned int) << 3) && (max >> exp) > 0
#define LOOP_MAX__(a, b)				\
    for(s = &vector[a], k = &vector[b]; s < k; ++s) {	\
	if(*s > max)  {					\
	    max = *s;					\
	}						\
    }
    \end{lstlisting}
    \begin{lstlisting}[caption=Алгоритм поразрядной сортировки]
    register unsigned int *b, *s, *k;
    register unsigned int exp = 0;
    register unsigned int max = exp;

    unsigned int i, *point[0x100];
    int swap = 0;
    
    const unsigned int preliminary = (size > PRELIMINARY__) ? PRELIMINARY__ : (size >> 3);
    

    LOOP_MAX__(1, preliminary);
    if(max <= (MAX_UINT__ >> 7)) {	
	LOOP_MAX__(preliminary, size);
    }
    
    b = (unsigned int *)malloc(sizeof(unsigned int) * size);
    
#define SORT_BYTE__(vec, bb, shift)					\
    unsigned int bucket[0x100] = {0};					\
    register unsigned char *n = (unsigned char *)(vec) + (exp >> 3),*m; \
    for(m = (unsigned char *)(&vec[size & 0xFFFFFFFC]); n < m;) {	\
	++bucket[*n]; n += sizeof(int);					\
	++bucket[*n]; n += sizeof(int);					\
	++bucket[*n]; n += sizeof(int);					\
	++bucket[*n]; n += sizeof(int);					\
    }									\
    for(n = (unsigned char *)(&vec[size & 0xFFFFFFFC]) + (exp >> 3),	\
	    m = (unsigned char *)(&vec[size]); n < m;) {		\
	++bucket[*n]; n += sizeof(int);					\
    }									\
    s = bb;								\
    int next = 0;							\
    for(i = 0; i < 0x100; ++i) {					\
	if(bucket[i] == size) {						\
	    next = 1;							\
	    break;							\
	}								\
    }									\
    if(next) {								\
	exp += 8;							\
	continue;							\
    }									\
    \end{lstlisting}
\begin{lstlisting}[caption=Алгоритм поразрядной
    for(i = 0; i < 0x100; s += bucket[i++]) {				\
	point[i] = s;							\
    }									\
    for(s = vec, k = &vec[size]; s < k; ++s) {	\

	*point[(*s shift) & 0xFF]++ = *s;				\
    }									\
    swap = 1 - swap;							\
    exp += 8;
    
    /* Sort each byte (if needed) */
    while(MISSING_BITS__) {
	if(exp) {
	    if(swap) {
		SORT_BYTE__(b, vector, >> exp);
	    } else {
		SORT_BYTE__(vector, b, >> exp);
	    }
	} else {
	    SORT_BYTE__(vector, b, );
	}
    }

    if(swap) {
	memcpy(vector, b, sizeof(unsigned int) * size);
    }
    
    free(b);
    
#undef PRELIMINARY__
#undef MISSING_BITS__
#undef LOOP_MAX__
#undef SORT_BYTE__
\end{lstlisting}

\begin{lstlisting}[caption=Алгоритм пирамидальной сортировки]
void max_heapify(int a[], int i, int n)
{
  int l,r,largest,loc;
  l=2*i;
  r=(2*i+1);
  if((l<=n)&&a[l]>a[i])
    largest=l;
  else
      \end{lstlisting}
\begin{lstlisting}[caption=Алгоритм пирамидальной сортировки]
    largest=i;
  if((r<=n)&&(a[r]>a[largest]))
    largest=r;
  if(largest!=i)
    {

      loc=a[i];
      a[i]=a[largest];
      a[largest]=loc;
      MAX_HEAPIFY(a, largest,n);
    }
}
void BUILD_MAX_HEAP(int a[], int n)
{
  for(int k = n/2; k >= 1; k--)
    {
      max_heapify(a, k, n);
    }
}
void heap_sort(int a[], int n)
{

  BUILD_MAX_HEAP(a,n);
  int i, temp;
  for (i = n; i >= 2; i--)
    {
      temp = a[i];
      a[i] = a[1];
      a[1] = temp;
      max_heapify(a, 1, i - 1);
    }
}
\end{lstlisting}
\newpage
\begin{lstlisting}[caption=алгоритм пузырьковой сортировки]
      void bubbleSort(vector <int> &v, int n) {
	bool swapped = true;
	int i = 0;

	while (i < n - 1 && swapped) {
		swapped = false;
		
		for (int j = n - 1; j > i; j--) { // unordered part
			
			if (v[j] < v[j - 1]) {
				swap(v[j], v[j - 1]);
				swapped = true;
			}
		}
		i++;
	}
}
\end{lstlisting}
 
\section{Вывод}

В данном разделе были разработаны исходный код всех трех алгоритмов: порразрядной сортировки, пирамидальной сортировки и сортировки пузырьком. 

%%% Local Variables:
%%% mode: latex
%%% TeX-master: "rpz"
%%% End:
